\documentclass[12pt, a4paper, oneside]{ctexart}
\usepackage{amsmath, amsthm, amssymb, graphicx}
\usepackage[bookmarks=true, colorlinks, citecolor=blue, linkcolor=black]{hyperref}
\usepackage{geometry}
\geometry{left=2.54cm, right=2.54cm, top=3.18cm, bottom=3.18cm}
\linespread{1.5}
\newtheorem{theorem}{theorem}[section]
\newtheorem{definition}[theorem]{definition}
\newtheorem{lemma}[theorem]{Lemma}
\newtheorem{corollary}[theorem]{推论}
\newtheorem{example}[theorem]{例}
\newtheorem{proposition}[theorem]{命题}
\newtheorem{remark}[theorem]{Remark}
% 导言区

\title{随想随感}
\author{Rogan}
\date{\today}

\begin{document}

\maketitle

\tableofcontents

\newpage
\section{chapter 1}

\begin{lemma}
    If a rectangle is the almost disjoint union of finitely many other rectangles, say $R=\bigcup _{k=1}^{N}R_k$, then$$|R|=\sum_{k=1}^N |R_k|$$

    If $R,R_1,. . . ,R_N $are rectangles, and $R\in \bigcup _{k=1}^{N}R_k$ ,then $$|R|\leq \sum_{k=1}^N |R_k|$$
\end{lemma}

\begin{remark}
    有限个几乎不交的并集,求面积可以直接相加;有交就小于等于。
\end{remark}

\begin{theorem}
    Every open subset $\mathcal{O} $ of $R$ can be writen uniquely as acountable union of disjoint open intervals
\end{theorem}

\begin{remark}
    任何开区间可以被分解为可数个不交的开区间的并集
\end{remark}

\begin{theorem}
    Every open subset $\mathcal{O} $ of $\mathbb{R} ^{d}$ ,$d\geq 1$, can be written as acountable union of almost disjoint closed cubes
\end{theorem}

\begin{remark}
    用小立方体(d维)去不断逼近
\end{remark}

\begin{definition}[Cantor Set]
    非空,有界,闭集,完全不连通,不可数集,零测
\end{definition}

\begin{definition}[Exterior Measure]
    
    if $E$ is any subset of $\mathbb{R} ^d$, the exterior measure of $E$ is $$m_{*}(E)=\inf \sum_{j=1}^{\infty}|Q_j|$$
\end{definition}

consedering any arbitrary covering of a closed cube $Q\in \bigcup_{j=1}^{\infty}Q_j$($Q_j$ is also closed cube), note that it suffices to prove that$$|Q|\leq \sum_{j=1}^{\infty}|Q_j|$$

稍微放大$Q_j$, 得到 open cube $S_j$, 之后可以利用有限覆盖定理, 得到 $$|Q|\leq (1+\epsilon)\sum_{j=1}^{M}|Q_j| \leq (1+\epsilon)\sum_{j=1}^{\infty}|Q_j|$$ So $$|Q|\leq m_{*}(Q)$$

对于 closed cube $Q$, $m_{*}(Q)\leq|Q|$显然, $|Q|\leq m_{*}(Q)$ 需要稍微放大$Q_j$成 open cube $S_j$, 之后用有限覆盖定理

对于 open cube $Q$, 取 closure of Q, 那么$m_{*}(Q)\leq|Q|$也同样显然,$|Q|\leq m_{*}(Q)$ 需要取任意被$Q$覆盖的 open cube $Q_0$, 而任何对于$Q$的立方体开覆盖必然覆盖$Q_0$, 所以下确界 $m_*(Q_)\geq |Q_0|$, 又可以取充分接近, 可得$|Q|\leq m_{*}(Q)$
\end{document}