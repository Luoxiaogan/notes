\documentclass[12pt, a4paper, oneside]{ctexart}
\usepackage{amsmath, amsthm, amssymb, graphicx}
\usepackage[bookmarks=true, colorlinks, citecolor=blue, linkcolor=black]{hyperref}
\usepackage{geometry}
\geometry{left=2.54cm, right=2.54cm, top=3.18cm, bottom=3.18cm}
\linespread{1.5}
\newtheorem{theorem}{定理}[section]
\newtheorem{definition}[theorem]{定义}
\newtheorem{lemma}[theorem]{引理}
\newtheorem{corollary}[theorem]{推论}
\newtheorem{example}[theorem]{例}
\newtheorem{proposition}[theorem]{命题}
% 导言区

\title{随想随感}
\author{Rogan}
\date{\today}

\begin{document}

\maketitle

\tableofcontents
\newpage
\section{随感}

\subsection{高代下期末复盘}

\date{2023年6月14日}

总算是结束了高代下的期末,开始复习普物和数分,现在来复盘一下今天上午的高代考试。

第一题是一个简单的计算题,$(a)$求出$f_{\phi}(\lambda)=d_{\phi}(\lambda)=(\lambda-2)^3$,$(b)$需要再找到矩阵表示与这组jordan基之间的关系即可,$(c)$由于只有一个jordan块且$\phi(X_1)=2X_1$,所以所有的不变子空间就是$0,<X_1>,<X_1,X_2>,<X_1,X_2,X_3>=\mathbb{R} ^3$,$(d)$最后转化到求jordan阵的n次幂,我最后一步算错了,现在看来可以考虑拆分成对角阵和幂零阵做二项式展开或许会好算很多。

第二题也是陈题稍稍改编之后的结果,只不过是把$W$由对角阵改成了对称阵,只需要注意在求标准正交基的时候记得$E_{ij}+E_{ji}(i<j)$还需要除以$\sqrt{2}$才行。

第三题也是陈题改编,将$W^{\bot }$定义到了$V^*$中,其余都完全一样。

第四题就是教材命题的默写,$d(\lambda)|f(\lambda)$用带余除法与最小多项式次数最小的性质加上$f(\phi)=0$即可解决,$(b)$要求证明$f(\lambda)$的根一定是$d(\lambda)$的根,只需要找到$\lambda_0$对应的特征向量$\alpha,\phi(\alpha)=\lambda_0\alpha$,然后证明$d(\lambda_0)(\alpha)=d(\phi)(\alpha)=0$即可。

第五题改编了前面的转置映射,将其换到了双曲空间的映射上面,但是总的方法还是各取双曲基之后表示成矩阵,利用展式研究两个矩阵之间的关系,这次比较有意思,并不是简单的转置那么简单,是分块变换了位置之后的转置,细心一点注意脚标即可。

第六题转化为矩阵语言之后就是极分解,我转化的时候表述成:一个可逆矩阵可以表示成一个正交矩阵右乘一个对角元均为正数的对角矩阵(实际上就是正定矩阵),考场上没有想起来,草草证明了一下子唯一性就到时间了。

复习一下极分解吧

\begin{theorem}[极分解]
    设$A$是$n$阶实方阵,则一定存在正交矩阵$Q$以及唯一的半正定矩阵$T$,使得$A=QT$,$A$可逆时,$Q$也是唯一的(此时$T$是正定的)
\end{theorem}
\begin{proof}
    存在性:由$A$的极分解,存在$Q_1,Q_2$为正交矩阵,使得
    $$ A=Q_1 \begin{pmatrix} \Delta  & 0 \\ 0 & 0 \end{pmatrix} Q_2=Q_1 Q_2 Q_2^{t}\begin{pmatrix} \Delta  & 0 \\ 0 & 0 \end{pmatrix}Q_2=QT$$
    其中$Q=Q_1 Q_2,T=Q_2^{t}\begin{pmatrix} \Delta  & 0 \\ 0 & 0 \end{pmatrix}Q_2$
    
    唯一性:设$A=QT=SP$,则$A^tA=T^2=P^2$,由于$A^tA$正定,所以$T=P=A^tA^P{\frac{1}{2}}$,若$A$可逆,则$T=S$可逆,则$Q=S$
\end{proof} 

总的来说,一题二题比较慌张,但是还是比较稳扎稳打的,三四五很快就写完了,五题的脚标让我纠结了半天,六题还是基础知识不牢固。
\newpage
\subsection{反思自己的学习习惯,学习心态,生活状态,人生目标,个人实现}

选择数学道路已有九个月有余,从顺利通过转系考是这一指标上来说,我的这一年可以是成功的,但现在回头反思,有许多地方都值得改进。

\subsubsection{努力}
大一上我选择不选本院系所有课程全选数院课程之后的两个月里都拼命内卷,但很多时候都浮于表面,没有落实到理解与思考上,这实际上是在用行动上的勤奋掩饰思维上的懒惰。

回想转系的初衷,既是因为未来发展的考量,也有对于化学学习与科研里面很多被戏称为“搬砖”的劳动力堆砌的不满足,总希望能去学习需要纯粹推理与深刻理解的知识。

大一上前半学期很多东西只是过了手,记住了它是什么,但是对于思想产生的intuition,逐步推导的动机,定理本身的直观理解,从证明方法中学到的常用的technique都没有进行深入思考。

所以说,演绎法(内化好最基本的知识,手法,构建自己的知识体系,深入理解清楚定义与定理)比归纳法(通过刷极多的题目来总结套路)优先级更高(二者并不对立,但是首先需要的是深刻的理解)

但我也有时沉迷于细节,只见树木不见森林,导致杂乱无章,不成体系,因此,好好预习,预习的时候着重思考与理解,好好上课,在手上过一遍,下来之后无需沉迷于细节,迅速开始做题,在做题过程中加深理解是最好的方法。

\subsubsection{交流}
交流有利于激发思维,很多自己想了和很久的问题和同学讨论之后都等加深理解。但要经过自己的深入思考,努力尝试之后再去交流才是最好的。

\subsubsection{思考}

除了记住它是什么,思想产生的intuition,逐步推导的动机,定理本身的直观理解,最基本的知识,手法,构建自己的知识体系,深入理解清楚定义与定理之外,统领这些的是对于思考的勤奋与坚韧。

想要深入理解,学会,而非只是知晓这些知识需要长时间,耐心的思考。长时间,耐心的思考是深入思考的必要条件,不要害怕麻烦。

学知识的时候努力去构建知识体系,画画图,多动手动脑,最后还是落实到动脑上面,计算与推导的最终目的还是在于构建脑海里面的知识图景,不应主次颠倒,我大一上就有很多东西全部记录在纸上却没有到脑子里面。

做题的时候努力去分析题目的含义,先构想直观的含义,在通览一下整体的思路(如果可以的话)。思路的来源可以是学习过程中常用的条件转化,直观的表示所导出的结果等。如果还是没有思路,去想想学过的知识点与技术,遍历一遍(这个时候就凸显出对于定义,定理,知识点,定理证明,技巧的内容与其使用环境的熟悉程度)。总之,不要轻易放弃。

\subsubsection{热情}

为什么可以持之以恒地做到深入地思考数学呢?我认为长久的坚持需要对于数学的热情与良好的学习习惯。

“数学好玩”我觉得是对这种态度的最好诠释,我不排斥做题,反而做题是对知识理解和学习检验的最好方法。我有幸进入北大数院,无论是去业界还是学界,在人这一生中最机敏,最富有灵性的十多年里与全中国最聪明的一群人一起交流与学习是非常难得的。不放沉浸其中,“We only live so long, we just want to do someting meaningful.”我希望自己是一个勇敢去拥抱可能性的人,努力去追赶与超越的人。

其中既有提高绩点的功利需求(想要在申请的时候绩点达到3.8),也有对于数学的深沉热爱。我爱的是理性,是纯粹为理解知识而行动的欢愉。

就像苏小糖学长在自述中写到的那样:很多朋友都谴责我“卷”,但我想说的是,我是真的喜欢数学,这不能算卷!我觉得就像是很多人喜欢看番一样的,从早看到晚那能叫卷吗?钻研数学的乐趣真的很让我陶醉,之前看pin的讲义经常三四个小时思考一道题目,有的时候一道题想出来了一看点,都晚上八点了只能去清青吃个快餐,甚至寒假回家生了个病,结果晚上做梦满脑子都在跑矩阵,醒来感觉脑子都快冒烟了。如果你对一件东西充满了热爱,我觉得就更容易将其做好。

深入理解与思考,没有热情的数学学习是无法持久的。

\subsubsection{生活习惯与自律}

早睡早起,刷牙洗脸,按时剪指甲,换床单,买好睡衣,平时的衣服裤子不要到床上去,勤换被单床单枕套,勤洗衣服。

自律不是对自己的命令,它来源于自治与自洽,来源于清醒的自我认知与自我行动。

对于手机的沉迷来源于平时生活的单调,所以需要统筹时间,积极参加活动,很多时候自己以为的不去参加活动而节省出来的时间实际上都被拿去
玩手机了,所以转念一下,与其玩手机,不如去参加活动,去看电影,去听戏曲,去跑步,去游泳,去交友,去聊天,去参加社团活动。

就像archer学长在他的微信里面写下的那样:努力运筹,努力生活。我的生活终究是属于自己的,我在世界上成长,我与他人交流,我努力去构建自己的生活,去实现自我与外界的平衡,不断前进。

\subsubsection{心态与未来}

我总是喜欢去浏览很多信息,担忧与大环境的未来,但实际上这是没有意义的,无论大环境怎么变化,无论我即将面临的是苦难还是辉煌,我都应该沉下心来,去体验,去经历,人生不过百年,去努力尝试各种可能性吧,保持心态的年轻,不是表面的幼稚,而是那不变的对一切事物的好奇与开放的心态。

\newpage
\subsection{学习地点与学习状态}

\date{2023年6月15日}

今天全天都在寝室学习,因为昨天熬夜到了三点半,今天上午十一点才爬起来,完成了普物的四五六章的知识学习和作业(第六章没有做完)。算是效率比较正常的一天,但是为了取得更好的期末成绩(考虑到普物和数分的期末都占比50\%)我需要更加地努力。

今天其实学习的时候还是没有做到专注于专心,也没有深入去思考与理解,看来思维的惰性的改正不是一朝一夕就可以完成的。

对于电介质极化和磁介质磁化的对比记忆很有用,但也遇到了一些问题,星期天晚上有普物三个小时的答疑时间,整理好问题去问老师吧。

分析数分也要提上日程,不能只去复习普物呀,普物4学分,数分5学分,需要付出更多的努力。

心儿渴望自由地飞翔,灵魂在知识的海洋里畅游。

“这真是一个圣地,梦中我来到这里,湖水泪水血水和汗水,告诉我这里没有游戏。”

不需要用自己作为一个北大的学生来自我绑架,但是我可以做的更好,加油!

\newpage

\subsection{数学学习的思想}

我先把命题写在一块黑板上,一张纸上,我的电脑里或一张表格上,之后试着用第一种方法展开论证。这一过程往往漫长而复杂,在整体验证所得结果的合理性和一致性后,我会将其分割成几个部分并加以记忆。接着,我在脑海中想象着同一张表格,一个接着一个地写出公式,对每个步骤进行必要的操作和计算——就像建筑架构那样。整个过程中我会锁定一些中间环节,他们如同坐标,时刻提醒我处于正确的轨道上 。有时我也会跳过一些我认为容易或可行的步骤。一旦完成了这项基本的脑力工作,部分或完整的架构就呼之欲出了。接着,我将对草稿上读出的各种要素加以说明,详细罗列脑海中每一个步骤所需的运算的操作。当然,大多数情况下我会发现其中的错误,它们有时能让我更好地理解问题,但无论如何,论证必须重新开始。有时甚至方法都需要重新尝试。(皮埃尔-路易$\cdot$利翁)

\newpage
\subsection{暑假学习}
学习内容:抽代,数分三,rudin,概率论,数算,

\newpage
\subsection{Life is a random walk}

\subsubsection{一}
转院尘埃落定后,我在微信上刷到了小糖哥哥去年写的转数文章,便也想写一篇来充当一个白鲸大学的样本点(隔壁来凑个热闹)。

我高中就读于某重点中学,初三暑假的时候学校要求去学一学竞赛,我在数学和物理之间摇摆,遂去询问年级主任,哪知他大手一挥,就把我叫去学了化学竞赛(理由是数学物理人太多了,去凑个数吧)。于是我就稀里糊涂地学了化学竞赛,好好看书,刷题,考试,进省队,考决赛,拿到降分,去了摆大。

拿到录取通知书之后,没开心几天,就开始担心自己会不会在大学被卷爆,于是赶紧下单高数线代c++开始学习(卷狗本质,别骂了)。看了差不多半本高数,我对于严谨性的要求确实无法容忍“证明略去”这样的文字,遂下单伍胜健老师的数分三册(也就是白鲸大学的数分三册小黄书,现在还是很庆幸当初买的不是很难的数分书),一边刷B站的网课,一边农药上分,最终赶在开学之前水完了数分前两册,而转数的想法也在潜意识中茁壮生长。

开学后,我有幸认识了原院系21届和20届转数的学长,经过了算不上理性的研判后,便all in选了三高(数分一,高代上,几何学),开始学习。我高中并没有接触过数学竞赛,见到满教室的金牌集训队大佬,仅仅用“鼠鼠”已经不足以形容我当初的不自信。在这样的压力下,我成功做到了“比高中更卷”,好好学习了两个月之后,我成功在数分和高代的期中考试上取得了自认为还不错的成绩(90+),但是在几何上寄了(现在想来,当初学的根本就不是解析几何计算,而是变换群的分类hhh)。

到了十一月,北京疫情爆发,很多同学回家了,我还是保持每天泡在图书馆里面学习,但一个人埋头学习的弊端也初步显现。我为了保持所谓的“卷度”,屏蔽了绝大部分的社交和玩耍,但大多数我节省出来的时间都被浪费在了做不出来题目的纠结与内耗中。死撑到了期末的我最终在考试周感染病毒和期末阴间题的组合拳下倒下了。

到了寒假,我仔细地反思了大一上的学习与生活,停止了机械化的内卷,开始有计划地进行社交和娱乐,并且找到了另一个转数的同学一起学习与讨论。我惊讶地发现自己的学习效率得到了极大的提升,并在寒假顺利地速通了一遍下学期的内容并把没做完的谢惠民上册写了一遍。

到了大一下开学,转数的学习小组从两人扩充至三个人,讨论内容也从转数拓展到了数算,学习方法,刷往年题,本研等。最终,小组里面三个人都有惊无险地通过了转数考试。

\subsubsection{二}
转数的心路历程其实也是我对于数学的认知过程。最开始学习的时候,我保持着高中生的学习方式:好好记下笔记,好好完成作业,认真思考问题,归纳总结题型,刷题准备考试。这样虽然能够取得较为不错的成绩,但是对于数学的认知还停留在形式逻辑的层级上。后来经过上学期几何学的教育,我认识到对于数学而言,很多时候不应该沉迷于细节,而应该去把握整体的脉络与框架,并建立一种潜意识上的直观感受,因为很多时候,证明的关键点在于像一个个坐标那样的idea,剩下的其实是技术性细节的填充。再后来,我与很多做本研的学长交流时,认识到自主学习并不应该局限于考试科目,更应该确定自己的兴趣与方向,尽早尽快地速通前置知识,在真正的科研中训练自己的直观与技术。于是,我的数学学习也终于从考试驱动转变到了兴趣驱动。

\subsubsection{三}

我的转数经历与感悟很多时候只是个人英雄史诗式的回顾,总的来说还是一帆风顺的,这主要得益于很多学长学姐与老师的帮助:

首先是之前转数群的前辈与同学们,我大一上刚刚开学就被群主(zhgg)拉入了23转数群(方圆百里来数院),在里面很多学长学姐解答了跨院系选课的操作与转数的流程。原院系转数成功的两位学长(hjgg和wkwgg)也在我陷入低谷时给予了至关重要的鼓励。archer学长一直作为我的榜样激励我在转数的道路上走下去。在群里我也认识了转数小组的两位好战友(bngg和zhgg),我们在四教小教室里讨论与交流的日子让我受益匪浅。

其次是原院系的老师与同学,他们对于我转院的想法给予了充分的包容与理解,我亲爱的室友们对我在寝室发电的行为(别急!)也给予了充分的容忍(当然很有可能时一起发电哈哈哈)。

之后是数院的授课老师们与助教gg们。zb老师虽然大一上期末数分出题过于阴间(鼠鼠我一个小时之后就动不了笔了)之外,是一位善良和蔼的老师。wfz老师同时照顾了我的绩点和饭点,大一上对于线性映射和对偶空间的讲解让我在大一下真正学习这些知识的时候轻松了很多。mx老师虽然薄纱了我的绩点,但是他认真严肃的教学风格,精心打磨的课程教案,高观点的几何课程令我感受到了真正的学者风范。tqc老师不仅成功拯救了我本学期的绩点,还经常关心我转数的进度,给了我许多鼓励。wlgg每次都把习题课的教案的Latex版本更新在网站上,让我这样不想去上习题课的摆烂人可以速成做题能力。lzygg在大一上的习题课很早地引入了线性映射的方法与很多数值计算的应用,令我受益良多。htzgg让我感受到了美妙的几何学,虽然我最后考寄了(不是助教gg的锅,是我太菜了),真的很对不起助教gg的付出呜呜呜。还有lygg和xygg,虽然我大一下的习题课基本上都被我用手抄教案替代了,但还是非常感谢认真负责的助教gg们!

\subsubsection{四}
未来呢?我会去做什么呢?最初决定转数院不仅有对严谨性的青睐,也有关于现实的考量,随着学习的深入,我愈发喜爱这个逻辑严密,精妙美丽的学科。我愿意在一生中最机敏的年纪与一群无比优秀的同龄人去领略数学的丛山峻岭,虽然我笨笨的,很多时候反应很迟钝,但是我也会尽力去学习我所好奇的知识。未来三年,希望能在sms过得快乐与充实!

\subsubsection{五}
关于转数的建议?好好学习,好好刷题,再做做转数考往年题,基本没有问题的!(考虑到大一大二加起来34:20的报录比,真的不卷!)

还有,快快加入24转数群!(方圆千里来数院!)

\newpage
\subsection{如何读数学教材}
记忆第一原理:记忆一个新知识的强弱程度由这个知识和旧知识的关联强弱决定。

举个例子,为什么一些人听课感觉自己懂了但是其实没懂,效果不如读书。因为听课的时候你压根没有那么多时间把新旧知识关联起来。读书虽然慢一点,但是你需要经常想这句话为什么对,需要跳回去去看。这样新旧知识就关联起来了。

好了,回到题主的问题。题主记不住的根本原因还是关联性不强,没有很好地把新知识和旧知识关联起来。换句白话:还是没理解。专业的东西记不住和记忆力没关系,是你专业不行的表现。举个例子,很多厉害的围棋手可以把一局棋从头摆到尾,看起来好像记忆力很厉害,但是他们到了日常生活貌似就没这种牛逼“记忆力”了。这是因为围棋在普通人和他们看来不是一回事。在这一点上,数学也是一样的。你老记不住的最大原因还是你没理解。

理解一个数学结果的程度可以分成好几个层次:

第一,只是看懂每句话,说服(骗)自己,自我认为懂了。

第二,自己可以机械重复证明。

第三,可以用“自己的语言体系”自然地解释一个结果。

第四,给出自己的证明。

从描述看来题主只是停留在第一和第二阶段,记不清是非常正常的事情。第一阶段相当于读一个文章保证读懂每句话而不去(努力)把握文章主旨和线索一样,这种情况读了基本等于没读。属于厕所读报纸的范畴。第二个略好,属于小学死记硬背的范畴。第三个阶段相当于读完文章去归纳主旨和段落大意,分析文章的结构。第四阶段等于不动手不舒服,要自己写篇文章去反驳和支持。你自己说,哪个记得清楚?我个人认为只有第三和第四属于“真的理解”这个境界。这两个的保持时间也最长, 基本可以到以“年”为计算单位。

由于大部分人长期并且坚定地处于前两个阶段,我就不讲它们了。直接谈第三和第四阶段。你能进入第三阶段的前提是你自己有一个“知识体系”,当然了,其实每个人都有,只是很多人不主动整理搭建。 每个数学的子学科都有一个基础骨架,每个人也有自己的一套理解系统,这两个东西耦合起来就是一个人的“语言体系”。白话地讲,就是一个人看到一个命题的时候“下意识地思考“。因为那个思考反映了“你内在的理解方式”。 而一个数学知识要让人很难忘记,最重要的原因是它完美符合了这个人“内在的理解方式”。 所以,为什么很多人喜欢“几何直观”,因为这个是几乎每个人最内化的理解方式(虽然不一定在任何情况下都是最好的)。

要做到“内化”一个知识,有很多方法。首先是丰富自己的骨架。举个例子,很多人对于“连续”的理解只喜欢通过“极限”而不能很好地用“开集”去理解,所以导致很多证明虽然他们能“读懂”,但是没做到内化。如果你能把连续的各种等价理解内化进你的知识骨架,那么此类问题也就不存在了。每当遇到让你觉得“困难”的证明的时候,你要设法去解析自己为什么觉得困难,是哪个过去的点你没能熟悉和内化。 当然了,由于每个人的数学系统多少有点“个性”,所以他们写的证明往往也只是“在他们的体系“中最自然。如果你觉得某个证明“过于奇葩”,你最好找个自己看得最舒服的证明方法。骨架完备后,你要做的是设法去重新解释一个定理。用画图也好,归纳也好,例子也罢。设法用“简单”的几句话概括。所谓费曼学习法的主旨也在于此。

每个人达成的方法不一样,给你个参考,我写明自己的方法。我的方法就是做电子笔记,因为它们便于阅读,不会丢失,容易修改。凡是重要的东西我都会多少做个笔记。这个笔记不是copy,这个笔记不是copy,这个笔记不是copy。而是comment 和remark(注释和评论)。 单纯复制是最蠢也最偷懒的做笔记方法,效果也最差。我完全不推荐。比如,我昨天看了个论文,其中有个结果由2个propositions,2个lemma,一个corollary构成。我自己把每个结果的自我完整理解写在自己的电子笔记中,同时自己概括后写了个remark放在(电子)笔记中。我打算今天再看看,继续简化和修正。我做的笔记我自己会经常翻阅和修正,这就是我的方法。 我知道某个数学大家的方法是做便签,他看过的论文和书都会贴上各种自己写的便签。这里再聊一个和我合作过的数学家,我把某个方法的讲授给他。然后我发现他压根不试图去copy这个思路,而是拿出一些简单的例子,自己写写画画,然后问了我一些问题。他和我说他在让这个方法真的“make sense”而且他试图给出一个新的证明。我和他合作的时候发现他随身携带本子,写得满满的,而且有时候会说“这个是新证明”,“我会在自己的论文里面加上这个新证明”。 其实,这是一种高等级的内化。 很可惜,国内很多人觉得“理解”等于“知道逻辑关系”而不是“内化知识“。

题主犯的一个错误就是认为“计算”靠“记忆”。 数学分析上的很多计算本质其实还是在看理解,你算不出来本质上你理解不了某些“计算方法”和“具体例子的特性”。这是(硬)分析不好的人的一个通病,他们能理解抽象命题但是把握不好一个具体的积分乃至函数。 唯一治疗的方法不是多做题,而是做完问题多思考,多去思考具体的例子。你缺乏的不是练习量,而是你思路有问题。

写几个注意的问题:

第一,不动脑子单纯机械默写证明和刷题很难让人进入第三和第四阶段。别太浪费时间做这些,你这样做还不如你静下心好好想诸如下面的问题:为什么这个定理需要“某个条件”,它在什么时候可以舍弃这个条件,它有几种证明方法?如果你想不出来就问别人和科学上网。

第二,多想想一个命题有什么“反例”也对你帮助不小。

第三,别憋着,有事情和别人商量商量,有时候你组织和论述一个问题的时候,自己就想明白了。第四,在数学这个问题上心急吃不了热豆腐。请接受别人的体系比你强大,长期内化习惯好造成的马太效应:内化习惯和能力越好的人学习速度越快,然后学习能力越快的人他们内化能力也越来越强,进入正循环。你要启动这个正循环,第一步不能瞄准人家的速度,而是培养内化的习惯。如果你没有内化的习惯,刚开始会非常慢非常难受。

\section{本科生科研}

\subsection{Zaiwen Wen}
Here is the relationship between directional derivative and subgradient for a convex function f(x) written in LaTeX:
\begin{align*}
f'(x;v) &= \lim_{t\to 0+} \frac{f(x+tv) - f(x)}{t} \\
\\
g \text{ is a subgradient of } f \text{ at } x &\Leftrightarrow f(y) \geq f(x) + g^T(y-x) \text{ for all } y
\\ \\
\text{If } g \text{ is a subgradient of } f \text{ at } x, \text{ then:} \\
g^Tv &\leq f'(x;v) \text{ for all directions } v
\end{align*}
Where:
- $f'(x;v)$ is the directional derivative of $f$ at $x$ in direction $v$
- $g$ is a subgradient of the convex function $f$ at point $x$
- The subgradient provides the tightest linear underestimator of $f$
- The directional derivative gives the slope along one specific direction $v$
- The subgradient slope $g^Tv$ must be less than the directional slope $f'(x;v)$
So in summary, the subgradient defines the sharpest possible linear underestimation of a convex function, while directional derivatives give slope along specific directions. The subgradient slope is always lower than or equal to the directional slope. (edited) 


\end{document}  