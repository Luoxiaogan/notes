\documentclass[12pt, a4paper, oneside]{ctexart}
\usepackage{amsmath, amsthm, amssymb, graphicx}
\usepackage[bookmarks=true, colorlinks, citecolor=blue, linkcolor=black]{hyperref}
\usepackage{geometry}
\geometry{left=2.54cm, right=2.54cm, top=3.18cm, bottom=3.18cm}
\linespread{1.5}
\newtheorem{theorem}{定理}[section]
\newtheorem{definition}[theorem]{定义}
\newtheorem{lemma}[theorem]{引理}
\newtheorem{corollary}[theorem]{推论}
\newtheorem{example}[theorem]{例}
\newtheorem{proposition}[theorem]{命题}
% 导言区

\title{随想随感}
\author{Rogan}
\date{\today}

\begin{document}

\maketitle

\tableofcontents

\section{随感}

\subsection{高代下期末复盘}

\date{2023年6月14日}

总算是结束了高代下的期末,开始复习普物和数分,现在来复盘一下今天上午的高代考试。

第一题是一个简单的计算题,$(a)$求出$f_{\phi}(\lambda)=d_{\phi}(\lambda)=(\lambda-2)^3$,$(b)$需要再找到矩阵表示与这组jordan基之间的关系即可,$(c)$由于只有一个jordan块且$\phi(X_1)=2X_1$,所以所有的不变子空间就是$0,<X_1>,<X_1,X_2>,<X_1,X_2,X_3>=\mathbb{R} ^3$,$(d)$最后转化到求jordan阵的n次幂,我最后一步算错了,现在看来可以考虑拆分成对角阵和幂零阵做二项式展开或许会好算很多。

第二题也是陈题稍稍改编之后的结果,只不过是把$W$由对角阵改成了对称阵,只需要注意在求标准正交基的时候记得$E_{ij}+E_{ji}(i<j)$还需要除以$\sqrt{2}$才行。

第三题也是陈题改编,将$W^{\bot }$定义到了$V^*$中,其余都完全一样。

第四题就是教材命题的默写,$d(\lambda)|f(\lambda)$用带余除法与最小多项式次数最小的性质加上$f(\phi)=0$即可解决,$(b)$要求证明$f(\lambda)$的根一定是$d(\lambda)$的根,只需要找到$\lambda_0$对应的特征向量$\alpha,\phi(\alpha)=\lambda_0\alpha$,然后证明$d(\lambda_0)(\alpha)=d(\phi)(\alpha)=0$即可。

第五题改编了前面的转置映射,将其换到了双曲空间的映射上面,但是总的方法还是各取双曲基之后表示成矩阵,利用展式研究两个矩阵之间的关系,这次比较有意思,并不是简单的转置那么简单,是分块变换了位置之后的转置,细心一点注意脚标即可。

第六题转化为矩阵语言之后就是极分解,我转化的时候表述成:一个可逆矩阵可以表示成一个正交矩阵右乘一个对角元均为正数的对角矩阵(实际上就是正定矩阵),考场上没有想起来,草草证明了一下子唯一性就到时间了。

复习一下极分解吧

\begin{theorem}[极分解]
    设$A$是$n$阶实方阵,则一定存在正交矩阵$Q$以及唯一的半正定矩阵$T$,使得$A=QT$,$A$可逆时,$Q$也是唯一的(此时$T$是正定的)
\end{theorem}
\begin{proof}
    存在性:由$A$的极分解,存在$Q_1,Q_2$为正交矩阵,使得
    $$ A=Q_1 \begin{pmatrix} \Delta  & 0 \\ 0 & 0 \end{pmatrix} Q_2=Q_1 Q_2 Q_2^{t}\begin{pmatrix} \Delta  & 0 \\ 0 & 0 \end{pmatrix}Q_2=QT$$
    其中$Q=Q_1 Q_2,T=Q_2^{t}\begin{pmatrix} \Delta  & 0 \\ 0 & 0 \end{pmatrix}Q_2$
    
    唯一性:设$A=QT=SP$,则$A^tA=T^2=P^2$,由于$A^tA$正定,所以$T=P=A^tA^P{\frac{1}{2}}$,若$A$可逆,则$T=S$可逆,则$Q=S$
\end{proof} 

总的来说,一题二题比较慌张,但是还是比较稳扎稳打的,三四五很快就写完了,五题的脚标让我纠结了半天,六题还是基础知识不牢固。

\subsection{反思自己的学习习惯,学习心态,生活状态,人生目标,个人实现}

选择数学道路已有九个月有余,从顺利通过转系考是这一指标上来说,我的这一年可以是成功的,但现在回头反思,有许多地方都值得改进。

\subsubsection{努力}
大一上我选择不选本院系所有课程全选数院课程之后的两个月里都拼命内卷,但很多时候都浮于表面,没有落实到理解与思考上,这实际上是在用行动上的勤奋掩饰思维上的懒惰。

回想转系的初衷,既是因为未来发展的考量,也有对于化学学习与科研里面很多被戏称为“搬砖”的劳动力堆砌的不满足,总希望能去学习需要纯粹推理与深刻理解的知识。

大一上前半学期很多东西只是过了手,记住了它是什么,但是对于思想产生的intuition,逐步推导的动机,定理本身的直观理解,从证明方法中学到的常用的technique都没有进行深入思考。

所以说,演绎法(内化好最基本的知识,手法,构建自己的知识体系,深入理解清楚定义与定理)比归纳法(通过刷极多的题目来总结套路)优先级更高(二者并不对立,但是首先需要的是深刻的理解)

但我也有时沉迷于细节,只见树木不见森林,导致杂乱无章,不成体系,因此,好好预习,预习的时候着重思考与理解,好好上课,在手上过一遍,下来之后无需沉迷于细节,迅速开始做题,在做题过程中加深理解是最好的方法。

\subsubsection{交流}
交流有利于激发思维,很多自己想了和很久的问题和同学讨论之后都等加深理解。但要经过自己的深入思考,努力尝试之后再去交流才是最好的。

\subsubsection{思考}

除了记住它是什么,思想产生的intuition,逐步推导的动机,定理本身的直观理解,最基本的知识,手法,构建自己的知识体系,深入理解清楚定义与定理之外,统领这些的是对于思考的勤奋与坚韧。

想要深入理解,学会,而非只是知晓这些知识需要长时间,耐心的思考。长时间,耐心的思考是深入思考的必要条件,不要害怕麻烦。

学知识的时候努力去构建知识体系,画画图,多动手动脑,最后还是落实到动脑上面,计算与推导的最终目的还是在于构建脑海里面的知识图景,不应主次颠倒,我大一上就有很多东西全部记录在纸上却没有到脑子里面。

做题的时候努力去分析题目的含义,先构想直观的含义,在通览一下整体的思路(如果可以的话)。思路的来源可以是学习过程中常用的条件转化,直观的表示所导出的结果等。如果还是没有思路,去想想学过的知识点与技术,遍历一遍(这个时候就凸显出对于定义,定理,知识点,定理证明,技巧的内容与其使用环境的熟悉程度)。总之,不要轻易放弃。

\subsubsection{热情}

为什么可以持之以恒地做到深入地思考数学呢?我认为长久的坚持需要对于数学的热情与良好的学习习惯。

“数学好玩”我觉得是对这种态度的最好诠释,我不排斥做题,反而做题是对知识理解和学习检验的最好方法。我有幸进入北大数院,无论是去业界还是学界,在人这一生中最机敏,最富有灵性的十多年里与全中国最聪明的一群人一起交流与学习是非常难得的。不放沉浸其中,“We only live so long, we just want to do someting meaningful.”我希望自己是一个勇敢去拥抱可能性的人,努力去追赶与超越的人。

其中既有提高绩点的功利需求(想要在申请的时候绩点达到3.8),也有对于数学的深沉热爱。我爱的是理性,是纯粹为理解知识而行动的欢愉。

就像苏小糖学长在自述中写到的那样:很多朋友都谴责我“卷”,但我想说的是,我是真的喜欢数学,这不能算卷!我觉得就像是很多人喜欢看番一样的,从早看到晚那能叫卷吗?钻研数学的乐趣真的很让我陶醉,之前看pin的讲义经常三四个小时思考一道题目,有的时候一道题想出来了一看点,都晚上八点了只能去清青吃个快餐,甚至寒假回家生了个病,结果晚上做梦满脑子都在跑矩阵,醒来感觉脑子都快冒烟了。如果你对一件东西充满了热爱,我觉得就更容易将其做好。

深入理解与思考,没有热情的数学学习是无法持久的。

\subsubsection{生活习惯与自律}

早睡早起,刷牙洗脸,按时剪指甲,换床单,买好睡衣,平时的衣服裤子不要到床上去,勤换被单床单枕套,勤洗衣服。

自律不是对自己的命令,它来源于自治与自洽,来源于清醒的自我认知与自我行动。

对于手机的沉迷来源于平时生活的单调,所以需要统筹时间,积极参加活动,很多时候自己以为的不去参加活动而节省出来的时间实际上都被拿去
玩手机了,所以转念一下,与其玩手机,不如去参加活动,去看电影,去听戏曲,去跑步,去游泳,去交友,去聊天,去参加社团活动。

就像archer学长在他的微信里面写下的那样:努力运筹,努力生活。我的生活终究是属于自己的,我在世界上成长,我与他人交流,我努力去构建自己的生活,去实现自我与外界的平衡,不断前进。

\subsubsection{心态与未来}

我总是喜欢去浏览很多信息,担忧与大环境的未来,但实际上这是没有意义的,无论大环境怎么变化,无论我即将面临的是苦难还是辉煌,我都应该沉下心来,去体验,去经历,人生不过百年,去努力尝试各种可能性吧,保持心态的年轻,不是表面的幼稚,而是那不变的对一切事物的好奇与开放的心态。



$$\int_{0}^{+\infty}\frac{\sin x}{x}dx=\frac{\pi}{2}$$
\end{document}