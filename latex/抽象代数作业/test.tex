\documentclass{report}

\input{preamble}
\input{macros}
\input{letterfonts}

\usepackage{ctex}

\title{\Huge{Abstract Algebra}\\Homework}
\author{\huge{Gan Luo}}
\date{2023.12.9}

\begin{document}

\maketitle
\newpage% or \cleardoublepage
% \pdfbookmark[<level>]{<title>}{<dest>}
\pdfbookmark[section]{\contentsname}{toc}
\tableofcontents
\pagebreak

\chapter{}
\section{homework 5a Part4}

\qs{Page 102, problem 17}{ $p$ is a prime, $p \equiv 1 \pmod 4$, prove that there exist $ a,b \in \mathbb{Z}$, such that $a^2+b^2=p$}

\sol
{

$p\equiv 1 \pmod 4$, so there exist $x, x^2 \equiv -1 \pmod p$, then $p \mid (x^2+1)$ in $\mathbb{Z}$, then $p \mid (x+i)(x-i)$ in $\mathbb{Z}[i]$, but $p \nmid (x+i), p \nmid (x-i)$, so $p$ is a pirme element in Euclidean domain $\mathbb{Z}[i]$, so p is reducibel in $\mathbb{Z}[i]$.

$\exists z_1,z_2 \in \mathbb{Z}[i], p=z_1z_2$, so let's cosider the norm of $p$, $N(p)=p^2=N(z_1)N(z_2)$, since $z\in \mathbb{Z}[i]$ is a unit(reversible) if and only if $N(z)=1$, $N(z_1)=N(z_2)=p$.

We have $z_1=a+bi$ with $a,b \neq 0$. And the statement that the norm of $z_1$ is $p$ is exactly the statement that $a^2+b^2=p$

So we have shown that $p \equiv 1 \pmod 4$ means that $p$ can be written as a sum of two squares (in a completely nonconstructive way). $\diamondsuit$

\begin{note}
    \begin{itemize}
        \item the norm of an element in $\mathbb{Z}[i]$ means $N(a+bi)=a^2+b^2$
        \item Euler's Creterion: $p$ is an odd prime, $a\in \mathbb{Z},(a,p)=1$\begin{align*}
                  a^{\frac{p-1}{2}} \equiv \begin{cases}1 &\pmod p \text{, if there exist an integer $x$ such that $x^2 \equiv a \pmod p$,}\\ -1 &\pmod p  \text{, if there is no such integer.} \end{cases}
              \end{align*}So since $p \equiv 1 \pmod 4$, we have $-1^{\frac{p-1}{2}}\equiv 1 \pmod p$,so there exist $x, x^2 \equiv -1 \pmod p$.
        \item $p$ is an odd prime. If $p \equiv 1 \pmod 4$, then $p$ is reducibel in $\mathbb{Z}[i]$. If $p \equiv 3 \pmod 4$, then $p$ is irreducibel in $\mathbb{Z}[i]$.
    \end{itemize}
\end{note}

}

\qs{Page 102, problem 18}{证明环$\mathbb{Z}[i]$的不可约元,在相伴意义下,只有以下三种:\\(1) $1+i$ ; (2) $a+bi, a,b \in \mathbb{Z}, a^2+b^2\equiv 1 \pmod 4 $为素数; (3) $p \equiv 3 \pmod 4$为素数.}

\sol
{

$\alpha \in \mathbb{Z}[i]$不可约,因此$\alpha$是素元,$\alpha \mathbb{Z}[i]$是素理想,$\alpha \mathbb{Z}[i]\cap \mathbb{Z}=(p)=p\mathbb{Z}$是$\mathbb{Z}$的素理想,因此$\alpha\mid p$.故$\alpha$不可约可以推出$\alpha$是素数在$\mathbb{Z}[i]$中的因子.

反之,若$\alpha \mid p$,由于$p$是有理素数,那么$\overline{\alpha} \mid p$,所以有$p=\alpha\overline{\alpha}r, r\in \mathbb{Z}[i]$, let's consider the norm of $p$, $N(p)=p^2=N(\alpha)N(\alpha)N(r)$,若$\alpha$非平凡,那么$N(\alpha)=p$, $p=\alpha\overline{\alpha}$, $N(\alpha)=p$,由于$\alpha$在$\mathbb{Z}[i]$中不可约.

因此,$\alpha \in \mathbb{Z}[i]$不可约 if and only if $\alpha$是素数$p$的非平凡因子.

$p=2=(1+i)(1-i)$, $i(1+i)=i-1=-(1-i), N(i)=1$,$1+i$与$1-i$在$\mathbb{Z}[i]$中相伴,$\alpha=1+i$.

$p \equiv 1 \pmod 4$, so there exist integer $a,b$, such that $a^2+b^2=p=(a+bi)(a-bi)$,故$\alpha=a+bi$.

$p \equiv 3 \pmod 4$,若存在$a,b \in \mathbb{Z}$, 使得$p=a^2+b^2$,根据下面的小定理,有$p \mid a$ and $p\mid b$, 因此矛盾,故$\alpha=p \equiv 3 \pmod 4$

\thm{}
{
    Let $p$ be a prime. If $p \equiv 3 \pmod 4$, $p \mid a^2+b^2$, then $p\mid a$ and $p\mid b$.
    
}

\begin{myproof}

    Using Fermats Little Theorem: $a^p \equiv a \pmod p, b^p \equiv b \pmod p$.

    Since $p \equiv 3 \pmod 4$, we have $a^{p+1} + b^{p+1} \equiv a^2+b^2 \equiv 0 \pmod p$. Because $4 \mid p+1$, we can write $p+1=4k$, so $a^{4k}+b^{4k} = a^{4k}+(b^2)^{2k} \equiv a^{4k}+(-a^2)^{2k}=2a^{4k} \pmod p$.
    
    由于$p \nmid 2$, $p \mid a^{4k}$, so $p \mid a$, 同理$p \mid b$.


\end{myproof}

}

\qs{5a-1}
{
    $F$ is a field, $R=\{f(x) \in F[x]| f(x)=a_0+\sum_{i=2}^n a_ix^i\}$. Prove that $R$是$F[x]$的子环; $x^2, x^3$是不可约元,但不是素元(so $R$ is not UFD).
}

\sol
{

    子环验证略.

    To prove that $x^2,x^3$ are irreducibel in $R$, just consider the $\deg$.

    $x^2,x^3$ are not prime, $x^2\mid x^3 \cdot x^3, x^2 \nmid x^3$ and $x^3 \mid x^2\cdot x^4, x^3 \nmid x^4, x^3\nmid x^2$
}

\qs{5a-2}
{
    $R$为UFD, $P$为$R$的非零素理想,证明:$P$中有素元.
}

\sol
{

    $P$ is nonzero, so $\exists a \in P, a\neq 0, a$ is irreversibel. Since $R$ is UFD, $a=a_1...a_n, a_i$ is irreducibel. Since $P$ is prime, $a_k \in P, k \in \{1,...,n\}$. Since $R$ is UFD, $a_k$ is prime.$\diamondsuit$
}


\begin{note}
    \begin{itemize}
        \item 诺特环的同态像是诺特环.
        \item (Hilbert基定理) $R$为交换诺特环, 那么$R[x]$为诺特环.
        \item 非UFD的诺特环: $\mathbb{Z}[\sqrt{-5}]$\\ $\mathbb{Z}$为PID, 故为诺特环,因此$\mathbb{Z}[x]$是诺特环,由于$\mathbb{Z}[\sqrt{-5}] \cong \mathbb{Z}/(x^2+5)$, 故$\mathbb{Z}[\sqrt{-5}]$是诺特环
        \item 非诺特环的UFD: $F[x_1,x_2,...,x_n,...]$
    \end{itemize}
\end{note}

\qs{5a-4}
{
    $R$ is UFD, $ab=c^n, a,b,c \in R^*, n \in \bbN_+$, $a,b$ are coprime, prove that there exist $u,v,f,g \in R$, $u,v$are invertibel, such that $a=uf^n, b=vg^n$.
}

\sol
{

    (i) If $a$ or $b$ is invertibel, WLOG, $a$ is invertibel, then $a=a\cdot 1^n, b=1 \cdot c^n$.

    (ii) If $a$ and $b$ are irreversibel, then $c^n$ is irreversibel, since $R$ is UFD, so $ab=(a_1...a_n)(b_1...b_m)=c^n=(c_1...c_t)^n$, where $a_i,b_j,c_s$ are irreducibel.

    使用相同的相伴代表元,由于$a,b$互素,因此没有不可逆的公因子,所以$a=ud_1^{e_1}...d_n^{e_n}, u$可逆,$b=vd_{n+1}^{e_{n+1}}...d_{n+s}^{e_{n+s}}, v$可逆,因此$a=uf^n,b=vg^n$.$\diamondsuit$
}

\qs{5a-5}{求$x^2+2=y^3$所有整数解.}

\sol
{

    $(x+\sqrt{-2})(x-\sqrt{-2})=y^3$ in $\mathbb{Z}[\sqrt{-2}]$.

    \begin{itemize}
            \item $\mathbb{Z}[\sqrt{-2}]$ is UFD.
            \item $x+\sqrt{-2}, x-\sqrt{-2}$无不可逆公因子
    \end{itemize}
    

    If $x+\sqrt{-2}=a_1...a_n, y=b_1...b_m$, then $x-\sqrt{-2}=\overline{a_1}...\overline{a_n}$, $a_i, b_j$ are irreducibel, since the fractorization is unique, $2n=3m$, so $n=3t, m=2t$.

    $x+\sqrt{-2},x-\sqrt{-2}$互素,因此,$x+\sqrt{-2}=(a+bi)^3=a^3-6ab+(3ab-2b^3)\sqrt{-2}$, then $b(3a-2b^2)=1$, so $b\in U(\mathbb{Z}[\sqrt{-2}])=\{1,-1\}$.

    $b=1$, then $a=1, x=-5,y=3$, or $a=-1, x=5, y=3$.

    $b=1$, no solution.

    So, all solutions are: $a=1, x=-5,y=3$, or $a=-1, x=5, y=3$.$\diamondsuit$

    \clm{}{}{$\mathbb{Z}[\sqrt{-2}]$ is UFD.}

    \begin{myproof}
        
        思路:证明$\mathbb{Z}[\sqrt{-2}]$是ED, 从而是UFD.

        $\forall \alpha, \beta \in \mathbb{Z}[\sqrt{-2}]$, $\alpha\beta^{-1}=u+v\sqrt{-2}, u,v \in \bbQ$, choose $a, b\in \bbZ, \alpha\beta^{-1}=u+v\sqrt{-2}= (a+b\sqrt{-2})+ [(u-a)+(v-b)\sqrt{-2}], |a-u| \leq \frac{1}{2}, |v-b| \leq \frac{1}{2}$.

        So $\alpha= \beta(a+b\sqrt{-2})+ \beta[(u-a)+(v-b)\sqrt{-2}]$, since $\alpha-\beta(a+b\sqrt{-2})= \beta[(u-a)+(v-b)\sqrt{-2}] \in \mathbb{Z}[\sqrt{-2}]$, let $q=a+b\sqrt{-2}, r= \beta[(u-a)+(v-b)\sqrt{-2}]  \in \mathbb{Z}[\sqrt{-2}]$, then $\alpha=\beta q +r, q,r \in \mathbb{Z}[\sqrt{-2}]$, $\delta(r)=N(r)=N(\beta)N((u-a)+(v-b)\sqrt{-2})=N(\beta)[(u-a)^2+2(v-b)^2]\leq N(\beta)\frac{3}{4}<N(\beta)$, so $\mathbb{Z}[\sqrt{-2}]$ is ED, thus UFD.$\diamondsuit$


    \end{myproof}

    \clm{}{}{$x+\sqrt{-2}, x-\sqrt{-2}$无不可逆公因子}

    \begin{myproof}
        
        若有$a\in \mathbb{Z}[\sqrt{-2}]$不可约,$a\mid x+\sqrt{-2}, x \mid x-\sqrt{-2}$,那么$a \mid 2\sqrt{-2}$.

        由于UFD中,不可约元是素元,所以$a \mid \sqrt{-2}, a= \pm \sqrt{-2}$, 但$\sqrt{-2} \nmid x+\sqrt{-2}$, 矛盾, 因此没有不可逆的公因子.$\diamondsuit$
    \end{myproof}
}

\qs{5a-6}{$R[x]$是PID$\iff R$是域.}

\sol
{

    ($\Rightarrow$): $R[x]$是PID, $x$在$R[x]$中不可约$\iff (x)$是极大理想$\Rightarrow R[x]/(x) \cong R$为域. 

    ($\Leftarrow$): $R$是域, 同高代方法.
    
}

\qs{5a-7}{$R$ is ED, prove that $\forall a\in R, a\neq 0$, a is invertibel $\iff \delta(a)= \min \delta(R^*)$  }

\sol
{

    ($\Rightarrow$): a is invertibel, $ab=1$, $\forall r\in R^*, r=(rb)a, \delta(a) \leq \delta (r)$.

    ($\Leftarrow$): $ \delta(a)= \min \delta(R^*)$, $1=aq+r, r=0, $ a is invertibel. 
}

\section{homework 6a Part1}

\qs{6a-1}{$K$是域$F$的代数扩域,$L$是$K$的包含$F$的子环,证明$L$是域}

\sol
{

    $L$是域$K$的子环,因此$L$是整环.

    $\forall s \in L\subset K, s \neq 0$, 因为$K$是$F$的代数扩域, 所以$s$在$F$上是代数的, 存在极小多项式$f(x)=a_nx^n+...+a_1x+a_0 \in F[x], f(s)=0$, 由于$f(x)$在$F[x]$上不可约,因此$a_0 \neq 0$, 所以$s(a_ns^{n-1}+...+a_1)(-a_0^{-1})=1, s^{-1}=(a_ns^{n-1}+...+a_1)(-a_0^{-1}) \in F \subset L $, 因此$L$是域.
    
}

\qs{6a-2}{$\alpha \in \mathbb{Q}(\sqrt[5]{3})\backslash \mathbb{Q}$, 证明$\sqrt[5]{3} \in \bbQ(\alpha)$}

\sol
{

    实际上,就是要证明: 如果$\alpha \in \mathbb{Q}(\sqrt[5]{3})\backslash \mathbb{Q}$, 那么$\bbQ(\sqrt[5]{3})=\bbQ(\alpha)$.

    可以巧妙地利用$5$是素数这一点.

    因为$\alpha \in \mathbb{Q}(\sqrt[5]{3})$, 所以$\bbQ(\alpha) \in \mathbb{Q}(\sqrt[5]{3})$. 因为$\alpha \in \mathbb{Q}(\sqrt[5]{3})\backslash \mathbb{Q}$, 所以$[\bbQ(\alpha): \bbQ]\geq2$. 而$[\mathbb{Q}(\sqrt[5]{3}): \bbQ]=5=[\mathbb{Q}(\sqrt[5]{3}): \bbQ(\alpha)][\bbQ(\alpha): \bbQ]$, 所以$[\bbQ(\alpha): \bbQ]=5, [\mathbb{Q}(\sqrt[5]{3}): \bbQ(\alpha)]=1$, 从而$\bbQ(\sqrt[5]{3})=\bbQ(\alpha), \sqrt[5]{3} \in \bbQ(\alpha)$.$\diamondsuit$
}

\qs{6a-3}{$K=\bbQ(\sqrt[3]{2}, e^{\frac{2\pi i}{3}})$, 给出$K$的子域$F$, $\alpha \in K$, 使得$[F:\bbQ]=3, [F(\alpha), \bbQ(\alpha)]=3$.}

\sol
{

    $F=\bbQ(\sqrt[3]{2}), [F:\bbQ]=\deg (x^3-2)=3$, $\alpha=e^{\frac{2\pi i}{3}}, [F(e^{\frac{2\pi i}{3}}): \bbQ(e^{\frac{2\pi i}{3}})]=[K:\bbQ(e^{\frac{2\pi i}{3}})]=3$
}

\qs{6a-5}{$a_1,...,a_n \in \bbN_+$, 两两互素, 都不是完全平方数, 证明: $[\bbQ(\sqrt{a_1},...,\sqrt{a_n}): \bbQ]=2^n$.}

\sol
{

    利用有限单扩张升链.

    $F_0=\bbQ, F_i=\bbQ(a_1,...,a_i)$, 考察$[F_{i+1}:F_i]$, 因为$a_1,...,a_n$两两互素, 因此$\sqrt{a_{i+1}} \neq F_{i}$, 因此$[F_{i+1}:F_i]>1$. 因为$a_{i+1}$不是完全平方数, $a_{i+1} \in \bbN_+ \subset \bbQ$, 所以$[F_{i+1}:F_i]=2$, 从而$[F_n:F_0]=2^n$.
}

\qs{6a-6}{$\alpha_1,...,\alpha_n \in \bbC, \alpha_i^2 \in \bbQ$, 证明: 域$\bbQ(\alpha_1,...,\alpha_n)$不包含$\sqrt[6]{2}$.}

\sol
{

    利用望远镜定理中的整除关系.

    $F_0=\bbQ, F_i=\bbQ(\alpha_1,...,\alpha_i)$, it's easy to show that $[F_{i+1}:F_i]=1$ or $2$.

    若$\bbQ(\alpha_1,...,\alpha_n)$包含$\sqrt[6]{2}$, 那么$[F_n:F_0]=2^k=[F_n:\bbQ(\sqrt[6]{2})][\bbQ(\sqrt[6]{2}):\bbQ], 6\mid 2^k$, 矛盾. 
}

\qs{6a-7}{证明: $\bbQ(\sqrt[3]{7}+2i)=\bbQ(\sqrt[3]{7},2i)$, 求$\sqrt[3]{7}+2i$在$\bbQ$上极小多项式.}

\sol
{

    显然有$\bbQ(\sqrt[3]{7}+2i)\subset \bbQ(\sqrt[3]{7},2i)$, 要证明: $\sqrt[3]{7},2i \in \bbQ(\sqrt[3]{7}+2i)$.

    $\alpha=\sqrt[3]{7}+2i, (\alpha-2i)^3=7, \alpha^3-12\alpha+(8-6\alpha^2)i=7, i=\frac{7-\alpha^3+12\alpha}{8-6\alpha^2} \in \bbQ(\sqrt[3]{7}+2i)$, also $\sqrt[3]{7} \in \bbQ(\sqrt[3]{7}+2i)$, then $\bbQ(\sqrt[3]{7}+2i)=\bbQ(\sqrt[3]{7},2i)$.(就是计算极小多项式的中间步骤)

    The degree of minimal polynomial of $\alpha$ over $\bbQ$: $\deg(f(x))=[\bbQ(\sqrt[3]{7}+2i):\bbQ]=6$.

    $f(x)=x^6+12x^4-13x^3+48x^2+168x+113$.$\diamondsuit$
}

\qs{6a-8}{域$E$是$\bbR$上的有限次扩张,证明$E=\bbR$或$E=\bbC$.}

\sol
{
    
    若$E=\bbR$, 那么$E=\bbR$.

    若$\exists a\in E \backslash \bbR, a\neq 0$, 那么$[\bbR(a):\bbR]>1$, 又因为$\bbR$上的不可约多项式$p(x)$次数为$2$, 因此$[\bbR(a):\bbR]=2$.

    由于$\bbC$是$\bbR$的代数闭包, 那么$p(x)$在$\bbC$中必定有根$\beta=c+di, d\neq 0$, 那么$\bbR(a)\subset E \subset \bbC$, $\bbR(a)=E=\bbC$.
}

\end{document}